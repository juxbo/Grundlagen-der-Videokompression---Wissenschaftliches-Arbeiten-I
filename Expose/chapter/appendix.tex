\section{Gliederungsentwurf}

\begin{enumerate}
\setlength\itemsep{-1em} % set spacing between bullet points
\item Einführung und Überblick
\item Grundlagen
\begin{enumerate}
	\setlength\itemsep{-1em} % set spacing between bullet points
	\item Irrelevanzreduktion
	\begin{itemize}
		\setlength\itemsep{-1em} % set spacing between bullet points
		\item Chroma Subsampling
		\item DCT
		\item Quantisierung
	\end{itemize}
	\item Redundanzreduktion
	\begin{itemize}
		\setlength\itemsep{-1em} % set spacing between bullet points
		\item Entropiecodierung
		\item Inter-, Intraprediction
		\item Motion Compensation
	\end{itemize}
\end{enumerate}
\item Fazit
\item Ausblick
\end{enumerate}

\section{Zeit- und Aufgabenplan}
\begin{table}[!h]
\begin{tabularx}{\textwidth}{ |c|X|c| }
  \hline
  \textbf{Woche} & \centering{\textbf{Aufgabe}} & \textbf{Meilenstein}  \\
  \hline 
  1   & Weitere Einarbeitung in das Thema Videokompression &  \\ \hline
  2   & Einarbeitung in Irrelevanzreduktion & Liste zu behandelnder Irrelevanzreduktionsmethoden \\ \hline
  3-4 & Zusammenfassung der zu behandelnden Irrelevanzreduktionsmethoden & Kapitel: Irrelevanzreduktion \\ \hline
  5   & Einarbeitung in Redundanzreduktion & Liste zu behandelnder Redundanzreduktionmethoden \\ \hline
  6   & Ausblick und Fazit erarbeiten & Kapitel: Ausblick und Fazit \\ \hline
  7   & Den erarbeiteten Überblick zusammenfassen & Kapitel: Einleitung und Überblick   \\ \hline
  8  & Korrektur lesen & Abschluss und Abgabe der Arbeit \\ \hline
\end{tabularx}
\end{table}

%\section{Vorläufiges Literaturverzeichnis}
%Überblick über bisher ermittelten Literaturquellen (Alphabetisch nach den Namen der Autoren sortiert)
%\nocite{*}
%\singlespacing % bibliography with single spacing 
%\bibliography{references}
