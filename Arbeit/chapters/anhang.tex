%\begin{appendices}
\appendix
\chapter{Weitere Abbildungen und Tabellen}
\label{appendix:abb_tab}
\begin{figure}[h!]
    \centering
    \includegraphics[scale=10]{images/2-1_chroma_artefacts_original.png}
    \includegraphics[scale=10]{images/2-1_chroma_artefacts_sampled.png}
    \caption{Artefakte durch Chroma Subsampling}
    \textit{Links: Original, Rechts: Subsampled. Die rechte Kante des blauen Farbblocks liegt in gesubsampleten 2x2 Blöcken, wodurch Artefakte entstehen. Die linke Kante liegt zwischen zwei 2x2 Blöcken, weshalb es zu keiner falschen Darstellung kommt.}
    \label{fig:chroma_artefacts}
\end{figure}

\vfill
\lstinputlisting[language=Python, caption=Implementierung des Quantisierungsprozesses nach MPEG-1 Standard ohne Clipping, label=lst:quantizer]{snippets/quantizer.py}
\vfill

\begin{figure}[h!]
    \centering
    \includegraphics[scale=0.5]{images/2-3_brook_orig.png}
    \includegraphics[scale=0.5]{images/2-3_brook_1.png}
    \includegraphics[scale=0.5]{images/2-3_brook_16.png}
    \includegraphics[scale=0.5]{images/2-3_brook_31.png}
    \caption{Ergebnis der Quantisierung mit verschiedenen Quantisierungsfaktoren}
    \textit{Oben links: Original, Oben rechts: Quantisiert mit Faktor 1, Unten links: Quantisiert mit Faktor 16, Unten rechts: Quantisiert mit Faktor 31.\\
    Mit zunehmendem Quantisierungsfaktor ist ein ansteigender Verlust der Bildqualität zu beobachten, wobei grobe Strukturen weitestgehend erhalten bleiben. Original nach \cite{brooke_cagle__2016}}
    \label{fig:quantization_multi_mquants}
\end{figure}


\begin{table}
\centering
\begin{tabular}{|c|c|c|c|c|c|c|c|}
	\hline
	8 & 16 & 19 & 22 & 26 & 27 & 29 & 34 \\
	16 & 16 & 22 & 24 & 27 & 29 & 34 & 37 \\
	19 & 22 & 26 & 27 & 29 & 34 & 34 & 38 \\
	22 & 22 & 26 & 27 & 29 & 34 & 37 & 40 \\
	22 & 26 & 27 & 29 & 32 & 35 & 40 & 48 \\
	26 & 27 & 29 & 32 & 35 & 40 & 48 & 58 \\
	26 & 27 & 29 & 34 & 38 & 46 & 56 & 69 \\
	27 & 29 & 35 & 38 & 46 & 56 & 69 & 83 \\
	\hline
\end{tabular}
\caption{Voreingestellte MPEG-1 Intracoding Quantisierungsmatrix. \cite{symes_peter_digital_2004} }
\label{tab:default_quant}
\end{table}

\begin{table}
\centering
\caption{Testergebnisse der angewandten Kompressionsalgorithmen}
\label{tab:test}
\begin{tabular}{cccll}
\multicolumn{3}{l}{Genutzte Optionen}               & Größe in Kilobyte & Ratio \\
RLE & Chroma Subsampling & Quantisierung mit Faktor &                   &       \\
X   &                    & -                        & 258.38            & 3.76  \\
X   & X                  & -                        & 145.95            & 6.66  \\
X   & X                  & 1                        & 58.32             & 16.67 \\
X   & X                  & 16                       & 18.59             & 52.29 \\
X   & X                  & 31                       & 16.23             & 59.89 \\
\end{tabular}
\end{table}

%\end{appendices}
