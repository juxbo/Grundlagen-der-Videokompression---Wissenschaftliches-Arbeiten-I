\chapter{Erläuterung des Testvorgehens}
\label{chap:testvorgenen}

Als Grundlage der Testvorgänge diente das dieser Arbeit auf CD beiliegende Programm. Das Chroma Subsampling ist als Mittelwertberechnung von jeweils 2x2 Blöcken während der Kompression realisiert. Die DCT ist wie im Kapitel \ref{chap:dct} vorgestellt implementiert. Als Quantisierungsmatrix wird die in Tabelle \ref{tab:default_quant} dargestellte MPEG-1 Intracoding Quantisierungsmatrix verwendet. Das Verfahren ist nach Listing \ref{lst:quantizer} implementiert. Als Lauflängencodierung wird eine abgewandelte Form des ZigZag Encodings nach JPEG-Standard verwendet. Hierbei werden lediglich Nullen lauflängencodiert, was in der Praxis ein Symbol für die Angabe des RLE encodierten Zeichens spart. Aufgrund der Charakteristik der aus der Quantisierung resultierenden Matrix hat sich diese Form als effizienter herausgestellt, als eine klassische RLE. Da kein Predictive Coding der DC-Werte verwendet wird, werden, anders als im JPEG Standard, auch diese Werte encodiert. Zur Berechnung der resultierenden Bildgrößen wurden folgende Werte angenommen: Jeder RGB Kanal lässt sich als 8 Bit Integer darstellen. Da jedes Pixel aus drei RGB Kanälen besteht resultiert hieraus eine Größe von 24 Bit. Zur Speicherung der Werte eines RLE encodierten Deskriptors werden aufgrund der zusätzlich benötigten Symbole 9 Bit benötigt. Als Quellbild wurde das in \ref{fig:quantization_multi_mquants} dargestellte Originalbild verwendet. Die Speicherung der RGB Werte benötigt nach oben beschriebener Annahme 324 Kilobyte. Das Bild hat eine Abmessung von 288x384 Pixel.
