\chapter{Irrelevanzreduktion}
\label{kap:Irrelevanzreduktion}

* Wie in Einleitung geschieben: Irrelevanzreduktion basiert auf Psychovisuellen Effekten. Im wesentlichen wird ausgenutzt:
	* Varianzen in der Helligkeit nimmt menschliches Auge besser wahr, als Varianzen im Farbton
	* Niedrige Ortsfrequenzen nimmt menschliches Auge besser wahr als hohe (Was zu Hölle sind Ortsfrequenzen?)
	* Quellen: Etwas Besseres als \cite{dankmeier_grundkurs_2006} S.358 \& S.359?
* pv. Effekte kann man sich so zu nutze machen, dass wir relevantere Informationen genauer Speichern, als weniger relevante
* Da Daten in RAW RGB im Normalfall vorliegen müssen wir die Daten erst vorbereiten, um sie dann nutzen zu können.
	* Um an Helligkeit heran zu kommen -> RGB -> YUV (Oder YCrCb, wo ist der Unterschied?)
		* Weiterverarbeitung via Subsampling
	* Um an Ortsfrequenzen heran zu kommen -> DCT
		* Weiterverarbeitung via Quantisierung
		* Wobei allerdings nicht direkt weniger Speicherplatz verbraucht wird, sondern vielmehr die Daten besser komprimierbar für RLE gemacht werden, welches im nächsten Kapitel behandelt wird.

\section{Chroma Subsampling}

* RGB -> YUV
* 4:2:0 -> 50\% Komprimierung!

\section{Diskrete Kosinus Transformation}

\section{Quantisierung}
