\chapter{Abstrakt}
Diese Arbeit gibt einen Überblick über die grundlegenden Methoden von Videokompressionsverfahren. Hierfür werden, sich am Encoding-Prozess von MPEG-1 orientierend, zunächst Arten der Irrelevanzreduktion, anschließend die wichtigsten Ansätze der Redundanzreduktion vorgestellt und anhand von Beispielcode erläutert. In eigenen Tests wurden unter Anwendung der vorgestellten Methoden zur verlustbehafteten und partieller Anwendung von verlustfreien Kompressionsalgorithmen Kompressionsraten mit einer Ratio von bis zu 1:37 erreicht. Diese Arbeit zeigt somit, wie mittels weniger Grundlagen bereits vergleichsweise hohe Einsparungen im Speicherverbrauch von Videos erreicht werden können. Insbesondere mit Blick auf die stetig steigenden Forderungen nach höheren Framerates, und besserer Auflösung wird deutlich, welch hohe Relevanz das Thema Videokompression auch in Zukunft haben wird.
