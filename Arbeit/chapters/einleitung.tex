\chapter{Einleitung} 

Videos sind seit der Entwicklung des Fernsehers zum Massenmedium kaum noch aus dem alltäglichen Leben wegzudenken. Seit dem Aufstieg des Internets als zentrales Kommunikationsmedium haben sich allerdings die Anforderungen an geeignete Speichertechniken von Videos drastisch verändert. Die heutigen Abspielgeräte haben noch immer begrenzten Speicherplatz und sind häufig nur mit schmalbandigen Internetanbindungen ausgestattet. Die Auflösung der Videos ist hingegen stark gestiegen. Um diese Ansprüche zu adressieren wurden Kompressionsalgorithmen entwickelt, die eine effiziente Speicherung speziell für bewegte Bilder ermöglichen. Die resultierenden Probleme aus dieser Art der Speicherung, wie Bildartefakte, sind heutigen Nutzern wohlbekannt. Die eigentliche Funktionsweise von Videokompression bleibt aber oft unbemerkt.

Deshalb möchten wir in dieser wissenschaftlichen Arbeit eine Übersicht über die Grundlagen von Videokompressionsverfahren geben.
