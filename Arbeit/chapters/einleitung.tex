\chapter{Einleitung} 

Videos sind seit der Entwicklung des Fernsehers zum Massenmedium kaum noch aus dem alltäglichen Leben wegzudenken. Seit dem Aufstieg des Internets als zentrales Kommunikationsmedium haben sich allerdings die Anforderungen an geeignete Speichertechniken von Videos drastisch verändert. Die heutigen Abspielgeräte haben noch immer begrenzten Speicherplatz und sind häufig nur mit schmalbandigen Internetanbindungen ausgestattet. Die Auflösung der Videos ist hingegen stark gestiegen. Um diese Ansprüche zu adressieren wurden Kompressionsalgorithmen entwickelt, die eine effiziente Speicherung speziell für bewegte Bilder ermöglichen. Die resultierenden Probleme aus dieser Art der Speicherung, wie Bildartefakte, sind heutigen Nutzern wohlbekannt. Die eigentliche Funktionsweise von Videokompression bleibt aber oft unbemerkt.

Diese Arbeit widmet sich den Grundlagen dieser Kompressionsverfahren. Zunächst werden die Methoden der Irrelevanzreduktion vorgestellt, die auf der Ausnutzung von psychovisuellen Effekten basieren. Hierbei werden gespeicherte Informationen entfernt, welche der menschliche Sehsinn nur schwer wahrnimmt und somit weniger relevant für das Erkennen des Bildes sind. Anschließend wird das Themenfeld der Redundanzreduktion beleuchtet, welche sich mit der Thematik beschäftigt, wie mehrmals vorkommende Informationen reduziert und entfernt werden können.

Diese Reihenfolge der Themen orientiert sich am MPEG-1 Encoding Prozess. MPEG-1 war einer der ersten relevanten digitalen Videokompressionsstandards, und wurde hauptsächlich für die Übertragung von digitalem Satellitenfernsehen eingesetzt. Im Vergleich zu aktuellen Standards sind die vorgeschlagenen Prozesse weniger komplex. Dieser Umstand macht MPEG-1 auch für die verwendeten Codebeispiele gut nutzbar.
