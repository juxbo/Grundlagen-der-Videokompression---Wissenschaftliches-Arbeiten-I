% ******************************
% *       Master Thesis        *
% *         Template           *
% *                            *
% *        FH Darmstadt        *
% *   Fachbereich Informatik   *
% ******************************

% scrbook (KOMA script) settings
\documentclass[liststotoc,bibtotoc,a4paper,12pt,parskip,final]{scrbook}
\usepackage[paper=a4paper,left=25mm,right=25mm,top=25mm,bottom=28mm,bindingoffset=0cm]{geometry}
\usepackage[utf8]{inputenc}
\usepackage[T1]{fontenc} % keine bitmap fonts
\usepackage{ae,aecompl} % truetype fonts
\usepackage[ngerman]{babel}
\usepackage{amssymb,amsmath}
\usepackage{mathtools}
\usepackage{graphicx}
\usepackage{url}
\usepackage{multicol} % Multicolumn layout
\usepackage{setspace}
\usepackage{color}
\usepackage{listings}
\usepackage[toc,page]{appendix}
\hbadness = 10000 % -> disable ``underfull \hbox'' warnings
\usepackage[pdftex,bookmarks=true]{hyperref} % -> automatische ``bookmarks''
\parindent 0pt % keine einr�ckung nach abs�tzen

% vspace, der vor beginn und nach Ende eines Chapters eingügt wird
\renewcommand*{\chapterheadstartvskip}{\vspace*{0cm}}
\renewcommand*{\chapterheadendvskip}{\vspace{1cm}}
% Graphics config
\DeclareGraphicsExtensions{.pdf,.jpeg,.png}

\usepackage[style=alphabetic]{biblatex}
\addbibresource{references.bib}

% Config listings
\lstdefinestyle{customc}{
  belowcaptionskip=1\baselineskip,
  breaklines=true,
  frame=L,
  xleftmargin=\parindent,
  language=C,
  showstringspaces=false,
%  basicstyle=\footnotesize\ttfamily,
  identifierstyle=\color{blue},
  stringstyle=\color{orange},
}

\lstdefinestyle{customasm}{
  belowcaptionskip=1\baselineskip,
  frame=L,
  xleftmargin=\parindent,
  language=[x86masm]Assembler,
  basicstyle=\footnotesize\ttfamily,
  commentstyle=\itshape\color{purple!40!black},
}

\lstset{escapechar=@,style=customc,captionpos=b}

\begin{document}
% 	\evensidemargin % breite seitenrand
% 	\oddsidemargin % breite seitenrand
	\frontmatter % keine kapitelnummern, r�mische seitenzahlen
	% % \addtocounter{page}{-1}

\begin{titlepage}
\begin{center}
\includegraphics[scale=1.5]{gfx/LG0_fbi_r5005_lzw.png}

\vspace{0.8cm}
% sans serif ANFANG
{\sf

{\LARGE\textbf{Hochschule Darmstadt}}\\ 

% {Large} ANFANG
{\Large - Fachbereich Informatik - \\ 

\vspace{2.0cm} 
{\Huge\textbf{Grundlagen der Videokompression}}\\ 

\vspace{2.0cm}
Seminararbeit im Kurs\\
\textbf{Wissenschaftliches Arbeiten in der Informatik I}\\ 

\vspace{1.0cm}
vorgelegt von\\ 
Justin Böhm und Matthias Greune

\vspace{1.0cm} 
\begin{tabular}{rl} 
Referent:& Michael Kröhn\\
\end{tabular}

\vspace{0.5cm}
\begin{tabular}{rl}
Ausgabedatum:& 21.10.2016\\
Abgabedatum:& 16.12.2016
\end{tabular}

} %{Large} ENDE

% sans serif ENDE
}

\end{center}
\end{titlepage}

	\onehalfspacing
	\chapter{Erklärung}
Ich versichere hiermit, dass ich die vorliegende Arbeit selbständig verfasst und keine anderen als die im Literaturverzeichnis angegebenen Quellen benutzt habe.
Alle Stellen, die wörtlich oder sinngemäß aus veröffentlichten oder noch nicht veröffentlichten Quellen entnommen sind, sind als solche kenntlich gemacht.
Die Zeichnungen oder Abbildungen in dieser Arbeit sind von mir selbst erstellt worden oder mit einem entsprechenden Quellennachweis versehen.
Diese Arbeit ist in gleicher oder ähnlicher Form noch bei keiner anderen
Prüfungsbehörde eingereicht worden.\\\\

Justin Böhm\\
Darmstadt, den \today\\[2cm]

Matthias Greune\\
Darmstadt, den \today\\

	\chapter{Abstrakt}
Diese Arbeit gibt einen Überblick über die grundlegenden Methoden von Videokompressionsverfahren. Hierfür werden, sich am Encoding-Prozess von MPEG-1 orientierend, zunächst Arten der Irrelevanzreduktion, anschließend die wichtigsten Ansätze der Redundanzreduktion vorgestellt und anhand von Beispielcode erläutert. In eigenen Tests wurden unter Anwendung der vorgestellten Methoden zur verlustbehafteten und partieller Anwendung von verlustfreien Kompressionsalgorithmen Kompressionsraten mit einer Ratio von bis zu 1:37 erreicht. Diese Arbeit zeigt somit, wie mittels weniger Grundlagen bereits vergleichsweise hohe Einsparungen im Speicherverbrauch von Videos erreicht werden können. Insbesondere mit Blick auf die stetig steigenden Forderungen nach höheren Framerates, und besserer Auflösung wird deutlich, welch hohe Relevanz das Thema Videokompression auch in Zukunft haben wird.

	\tableofcontents
	\listoffigures
	\mainmatter
	\chapter{Einleitung} 

Videos sind seit der Entwicklung des Fernsehers zum Massenmedium kaum noch aus dem alltäglichen Leben wegzudenken. Seit dem Aufstieg des Internets als zentrales Kommunikationsmedium haben sich allerdings die Anforderungen an geeignete Speichertechniken von Videos drastisch verändert. Die heutigen Abspielgeräte haben noch immer begrenzten Speicherplatz und sind häufig nur mit schmalbandigen Internetanbindungen ausgestattet. Die Auflösung der Videos ist hingegen stark gestiegen. Um diese Ansprüche zu adressieren wurden Kompressionsalgorithmen entwickelt, die eine effiziente Speicherung speziell für bewegte Bilder ermöglichen. Die resultierenden Probleme aus dieser Art der Speicherung, wie Bildartefakte, sind heutigen Nutzern wohlbekannt. Die eigentliche Funktionsweise von Videokompression bleibt aber oft unbemerkt.

Diese Arbeit widmet sich den Grundlagen dieser Kompressionsverfahren. Zunächst werden die Methoden der Irrelevanzreduktion vorgestellt, die auf der Ausnutzung von psychovisuellen Effekten basieren. Hierbei werden gespeicherte Informationen entfernt, welche der menschliche Sehsinn nur schwer wahrnimmt und somit weniger relevant für das Erkennen des Bildes sind. Anschließend wird das Themenfeld der Redundanzreduktion beleuchtet, welche sich mit der Thematik beschäftigt, wie mehrmals vorkommende Informationen reduziert und entfernt werden können.

Diese Reihenfolge der Themen orientiert sich am MPEG-1 Encoding Prozess. MPEG-1 war einer der ersten relevanten digitalen Videokompressionsstandards, und wurde hauptsächlich für die Übertragung von digitalem Satellitenfernsehen eingesetzt. Im Vergleich zu aktuellen Standards sind die vorgeschlagenen Prozesse weniger komplex. Dieser Umstand macht MPEG-1 auch für die verwendeten Codebeispiele gut nutzbar.

	\chapter{Irrelevanzreduktion}
\label{kap:Irrelevanzreduktion}

* Wie in Einleitung geschieben: Irrelevanzreduktion basiert auf Psychovisuellen Effekten. Im wesentlichen wird ausgenutzt:
	* Varianzen in der Helligkeit nimmt menschliches Auge besser wahr, als Varianzen im Farbton
	* Niedrige Ortsfrequenzen nimmt menschliches Auge besser wahr als hohe (Was zu Hölle sind Ortsfrequenzen?)
	* Quellen: Etwas Besseres als \cite{dankmeier_grundkurs_2006} S.358 \& S.359? -> Vllt \cite{akramullah_digital_2014} S.13
* pv. Effekte kann man sich so zu nutze machen, dass wir relevantere Informationen genauer Speichern, als weniger relevante
* Da Daten in RAW RGB im Normalfall vorliegen müssen wir die Daten erst vorbereiten, um sie dann nutzen zu können.
	* Um an Helligkeit heran zu kommen -> RGB -> YUV (Oder YCrCb, wo ist der Unterschied?)
		* Weiterverarbeitung via Subsampling
	* Um an Ortsfrequenzen heran zu kommen -> DCT
		* Weiterverarbeitung via Quantisierung
		* Wobei allerdings nicht direkt weniger Speicherplatz verbraucht wird, sondern vielmehr die Daten besser komprimierbar für RLE gemacht werden, welches im nächsten Kapitel behandelt wird.

\section{Chroma Subsampling}

* RGB -> YUV
* Formeln nach https://www.fourcc.org/fccyvrgb.php --> Gibt es vllt. eine bessere Quelle? Kann ich das nach CCIR 601 ableiten?
* 4:2:0 -> 50\% Komprimierung!

\section{Diskrete Kosinus Transformation}

* DCT ist eine spezielle Form der Fourier-Transformation
	* Fourier-Transformation aproximiert eine Funktion mittels Sinus-Funktionen
	* 4 Probleme \cite{symes_peter_digital_2004} S.71:
		* *It assumes that the time domain signal is infinite in extent*
		* *It assumes continous funtions in time*
		* Nicht ohne weiteres auf 2D anwendbar
		* Generierte Koeffitienten sind 2D (Amplitude + Phase bzw. sinus + cosine)
	* DCT funktioniert, solange nach dem Nyquist Theorem gesampled wurde (warum?)
	* Nutzt außerdem noch einen Effekt aus, an den ich mich gerade nicht mehr erinnere ->bandwidth-limited data
* DCT erlaubt uns Ortsfrequenzen zu extrahieren (warum? wodurch?)

* Formel: \[F(u,v) = \frac{1}{4} C_uC_v\sum_{x=0}^7 \sum_{y=0}^7 f(x,y) \cos \left(\frac{(2x+1)u\pi}{16}\right) \cos\left(\frac{(2y+1)v\pi}{16}\right) \]
	* Quelle \cite{symes_peter_digital_2004} S.75
* Implementierung: Siehe src/dct.py

* Es wird eine zweidimensionale DCT verwendet.

* Wann funktioniert sie nicht so gut?

\section{Quantisierung}

	\chapter{Redundanzreduktion}
\label{kap:Redundanzreduktion}

Die Redundanzreduktion ist cool und sie reduziert Redundanz.

\section{Entropiecodierung}

\section{Motion Compensation}

Alle bis jetzt vorgestellten Ansätze der Videokompression beschäftigen sich mit der Kompression von Einzelbildern innerhalb eines Videos. Bei der Motion Compensation hingegen wird das Kompressionspotential ausgenutzt, dass innerhalb der Abhängigkeiten der Einzelbilder in einem Video steckt.
Videos bestehen meist aus zusammenhängenden Szenen mit größtenteils unverändertem Inhalt innerhalb einer jeweils solchen Szene.

Man stelle sich zum Beispiel die folgende Szene vor: Eine statische Kamera filmt einen Mensch, unseren Protagonisten, der eine Straße entlang läuft und schließlich eine Bar betritt, eine typische Szene in Serien heutzutage.

Teilt man diese Szene in ihre Einzelbilder auf, stellt man schnell fest, dass die Einzige Bewegung der laufende Protagonist ist und der Hintergrund dabei komplett statisch verbleibt.
Motion Compensation nutzt die Redundanz dieser statischen Hintergründe aus indem es diese jeweils nur ein Mal speichert und in den folgenden Bildern darauf referenziert um ein für den Zuschauer unverändertes Bild anzuzeigen.
Da Videos üblicherweise zu großen Teilen mit statischen Bildteilen übersäht sind, macht die von Motion Compensation erzielbare Kompression einen großen Teil des gesamt möglichen Kompressionpotentials innerhalb von Videos aus.

Damit Motion Compensation überhaupt funktionieren kann ist eine Aufteilung und Auswertung aller Video Einzelbilder (Frames) nötig.
\subsection{Frames}
Mit dem Kodieren teilt man alle Frames in eine bestimmte Bildart ein:
Es gibt rein intracodierte Frames, die sogenannten intracoded Frames (kurz I-Frames), bei denen es sich um einzelne Vollbilder, die Allein stehen und somit von keinem anderen Bild des Videos abhängen. Bei ihnen handelt es sich im Endeffekt einfach um ein für sich stehendes JPEG, was mit den üblichen Methoden der Bildkompression verkleinert wurde.
Außerdem gibt es intercodierte Frames, die nur einen vorhergesagte Differenz des Inhalt in Abhängigkeit zu einem vorherigen I-Frame haben, die sogenannten predictive Frames (kurz P-Frames).
Als letztes gibt es bipredictive Frames (kurz B-Frames), die sehr ähnlich zu P-Frames, die in zwei Richtungen intercodiert sind, nämlich indem sie die vorhergesagte Differenz des Inhaltes zum vorherigen I- oder P-Frame speichern.
Um die Vorhersagung zu erreichen zu können wird eine Reihenfolge der Codierung gewählt, die ungleich der Reihenfolge der Anzeigereihenfolge ist, wie auf der Abbildung X erkennbar ist. Dadurch wird der sowieso schon komplexe Prozess zusätzlich erschwert.

Ein kompletter Szenenwechsel, also das Ändern des kompletten Bildes, ohne statische Zusammenhänge, muss dem Encoder immer mitgeteilt werden. Dieser muss dann einen neuen I-Frame codieren, auf dem die folgenden P- und B-Frames basieren. Dadurch wird die potentielle Gefahr einer starken Artefaktbildung vorgebeugt.

Wenn man diese Aufteilung jedoch jeweils nur einmal pro Szene anwenden würde, würden mehrere Probleme bei wahllosem Zugriff entstehen. Wenn der I-Frame einer Szene fehlt oder übersprungen wird, würden die Änderungen, die in den folgenden P- und B-Frames festgehalten wurden, auf den falschen I-Frame angewendet, sodass im Video starke Artefakte entstehen. Beim Ausfall eines P-Frames einer Szene würde grundsätzlich das Gleiche gleiche passieren, jedoch nur bei den noch folgenden P-Frames der Szene.

Um diese unschönen Artefakte beim Vor- und Zurückspulen zu verhindern, dürfte nur zu einem I-Frame gesprungen werden, welches bei einer Aufteilung pro Szene jeweils der Anfang einer neuen Szene wäre.

Da bei einem Großteil der Anwendungsfälle von Videos jedoch eine fast vollständig wahlfreier Zugriff gewünscht ist, teilt man sie in viele kleine aufeinanderfallende Bildergruppen (Group of pictures, kurz GOP) auf. Eine GOP wird meist mit 2 Parametern angegeben, zum Beispiel N und M.
Dabei ist N eine bestimmte Anzahl von Frames aus denen die GOP besteht, also die Distanz von einem I-Frame zum nächsten I-Frame.
M gibt die Distanz von einem I- oder P-Frame, bis zum jeweils Nächsten an, somit ist M-1 die Anzahl von B-Frames, die nach einem I- oder P-Frame folgen. Eine Bildergruppe fängt immer mit einem I-Frame an und wiederholt sich bis zum Ende eines Videos mit einem konstanten Schema.

Mit den Parametern N=12 und M=4, würde die GOP dann aussehen wie auf der Abbildung X. (IBBBPBBBPBBB I...) TODO: ABBILDUNG 

Bei MPEG ist eine Aufteilung mit den Parametern M=3 bis 4 und N= 11 bis 15 üblich.

Betrachtet man eine übliche Framerate von 25 ist somit wahlfreier Zugriff bis auf die Hälfte einer Sekunde gegeben. Außerdem wird dadurch bei leichten Übertragungsfehlern einer Videodatei der Schaden minimiert, sodass das vom Endnutzer gesehene Bild nur maximal eine halbe Sekunde Artefakte anzeigt.

\subsection{Makroblocks}
TODO:
Jeder Intercodierte Frame wird in sogenannte Makroblöcke unterteilt. Diese Makroblöcke werden beim Codieren zum Vergleichen mit dem vorher codierten Bild mittels Block matching algorithmus benutzt.
\subsection{Motion Estimation and Compensation}: 
TODO:
Wenn ein ähnlicher Block gefunden wird, wird der Block mittels dem resultierenden Motion Vektor encodiert.
	\chapter{Ausblick}
\label{kap:Ausblick}
ÄÖÜäöüß
	\chapter{Zusammenfassung}
\label{kap:Zusammenfassung}
ÄÖÜäöüß
	\setcounter{page}{15} % seitenzahl counter auf 15 setzen
% 	\include{glossar}
	\appendix
\chapter{Weitere Abbildungen und Tabellen}
\label{appendix:abb_tab}
\begin{figure}[h!]
    \centering
    \includegraphics[scale=10]{images/2-1_chroma_artefacts_original.png}
    \includegraphics[scale=10]{images/2-1_chroma_artefacts_sampled.png}
    \caption{Artefakte durch Chroma Subsampling}
    \textit{Links: Original, Rechts: Subsampled. Die rechte Kante des blauen Farbblocks liegt in gesubsampleten 2x2 Blöcken, wodurch Artefakte entstehen. Die linke Kante liegt zwischen zwei 2x2 Blöcken, weshalb es zu keiner falschen Darstellung kommt.}
    \label{fig:chroma_artefacts}
\end{figure}

\begin{figure}[h!]
    \centering
    \includegraphics[scale=0.58]{images/2-3_brook_orig.png}
    \includegraphics[scale=0.58]{images/2-3_brook_1.png}
    \includegraphics[scale=0.58]{images/2-3_brook_16.png}
    \includegraphics[scale=0.58]{images/2-3_brook_31.png}
    \caption{Ergebnis der Quantisierung mit verschiedenen Quantisierungsfaktoren}
    \textit{Oben links: Original, Oben rechts: Quantisiert mit Faktor 1, Unten links: Quantisiert mit Faktor 16, Unten rechts: Quantisiert mit Faktor 31.\\
    Mit zunehmendem Quantisierungsfaktor ist ein ansteigender Verlust der Bildqualität zu beobachten, wobei grobe Strukturen weitestgehend erhalten bleiben. Original nach \cite{brooke_cagle__2016}}
    \label{fig:quantization_multi_mquants}
\end{figure}


\begin{table}
\centering
\begin{tabular}{|c|c|c|c|c|c|c|c|}
	\hline
	8 & 16 & 19 & 22 & 26 & 27 & 29 & 34 \\
	16 & 16 & 22 & 24 & 27 & 29 & 34 & 37 \\
	19 & 22 & 26 & 27 & 29 & 34 & 34 & 38 \\
	22 & 22 & 26 & 27 & 29 & 34 & 37 & 40 \\
	22 & 26 & 27 & 29 & 32 & 35 & 40 & 48 \\
	26 & 27 & 29 & 32 & 35 & 40 & 48 & 58 \\
	26 & 27 & 29 & 34 & 38 & 46 & 56 & 69 \\
	27 & 29 & 35 & 38 & 46 & 56 & 69 & 83 \\
	\hline
\end{tabular}
\caption{Voreingestellte MPEG-1 Intracoding Quantisierungsmatrix. \cite{symes_peter_digital_2004} }
\label{tab:default_quant}
\end{table}

\begin{table}
\centering
\caption{Testergebnisse der angewandten Kompressionsalgorithmen}
\label{tab:test}
\begin{tabular}{cccll}
\multicolumn{3}{l}{Genutzte Optionen}               & Größe in Kilobyte & Ratio \\
RLE & Chroma Subsampling & Quantisierung mit Faktor &                   &       \\
X   &                    & -                        & 258.38            & 3.76  \\
X   & X                  & -                        & 145.95            & 6.66  \\
X   & X                  & 1                        & 58.32             & 16.67 \\
X   & X                  & 16                       & 18.59             & 52.29 \\
X   & X                  & 31                       & 16.23             & 59.89 \\
\end{tabular}
\end{table}


\begin{figure}[h!]
    \centering
    \includegraphics[scale=0.566]{images/corruptedGif/frame7.png}
    \includegraphics[scale=0.566]{images/corruptedGif/frame10.png}
    \includegraphics[scale=0.566]{images/corruptedGif/frame13.png}
    \includegraphics[scale=0.566]{images/corruptedGif/frame16.png}
    \includegraphics[scale=0.566]{images/corruptedGif/frame19.png}
    \includegraphics[scale=0.566]{images/corruptedGif/frame22.png}
    \caption{Bildartefakte beim Auslassen eines I-Frames}
    \textit{Es wird jeder dritte Frame eines GIFs gezeigt, welches einen Szenenwechsel von einem Turner hinzu zu zwei springendem Delfinen darstellt. Hierbei wurde im Ursprungsvideo mit Absicht der beim Szenenwechsel erzeugte I-Frame zerstört um künstlich Bildartefakte zu erzwingen und sogenannte „Glitch-Art” zu erzeugen.}
    \linebreak
    Original nach \cite{supergif}
\end{figure}

\begin{figure}[h!]
    \centering
    \includegraphics[scale=3]{images/3-2-3_motionCompensation.jpg}
    \caption{Suchen von identischen Blöcken in zwei Frames mittels Bewegungskorrektur}
    \textit{Das Referenz Bild wird anschließend mit dem resultierenden Motion Vektor codiert}
    Quelle: \cite{lopes_memory_2012}
\end{figure}

\chapter{Listings}

\lstinputlisting[language=Python, caption=Implementierung der DCT für ein 8x8 Array, label=lst:impl_dct]{snippets/dct.py}
\newpage
\lstinputlisting[language=Python, caption=Implementierung des Quantisierungsprozesses nach MPEG-1 Standard ohne Clipping, label=lst:quantizer]{snippets/quantizer.py}

\chapter{Erläuterung des Testvorgehens}
\label{chap:testvorgenen}

Als Grundlage der Testvorgänge diente das dieser Arbeit auf CD beiliegende Programm. Das Chroma Subsampling ist als Mittelwertberechnung von jeweils 2x2 Blöcken während der Kompression realisiert. Die DCT ist wie im Kapitel \ref{chap:dct} vorgestellt implementiert. Als Quantisierungsmatrix wird die in Tabelle \ref{tab:default_quant} dargestellte MPEG-1 Intracoding Quantisierungsmatrix verwendet. Das Verfahren ist nach Listing \ref{lst:quantizer} implementiert. Als Lauflängencodierung wird eine abgewandelte Form des ZigZag Encodings nach JPEG-Standard verwendet. Hierbei werden lediglich Nullen lauflängencodiert, was in der Praxis ein Symbol für die Angabe des RLE encodierten Zeichens spart. Aufgrund der Charakteristik der aus der Quantisierung resultierenden Matrix hat sich diese Form als effizienter herausgestellt, als eine klassische RLE. Da kein Predictive Coding der DC-Werte verwendet wird, werden, anders als im JPEG Standard, auch diese Werte encodiert. Zur Berechnung der resultierenden Bildgrößen wurden folgende Werte angenommen: Jeder RGB Kanal lässt sich als 8 Bit Integer darstellen. Da jedes Pixel aus drei RGB Kanälen besteht resultiert hieraus eine Größe von 24 Bit. Zur Speicherung der Werte eines RLE encodierten Deskriptors werden aufgrund der zusätzlich benötigten Symbole jeweils 9 Bit benötigt. Als Quellbild wurde das in \ref{fig:quantization_multi_mquants} dargestellte Originalbild verwendet. Die Speicherung der RGB Werte benötigt nach oben beschriebener Annahme 324 Kilobyte. Das Bild hat eine Abmessung von 288x384 Pixel.


	\backmatter % keine kapitelnummern, r�mische seitenzahlen
	% \bibliographystyle{alpha}
	% \nocite{*}
	% \bibliography{references}
	\printbibliography
\end{document}
